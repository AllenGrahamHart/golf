\documentclass[11pt,a4paper]{article}

\usepackage[margin=1in]{geometry}
\usepackage{graphicx}
\usepackage{amsmath,amssymb}
\usepackage{booktabs}
\usepackage{hyperref}
\usepackage{subcaption}
\usepackage{enumitem}

\title{\textbf{AI Golf Coach: Automated Swing Analysis\\Using 3D Human Pose Estimation}}
\author{Allen Hart, Tom Montenegro-Johnson}
\date{}

\begin{document}

\maketitle

\section{Executive Summary}

We propose to develop a proof-of-concept AI golf coach that provides automated, personalized feedback on golf swings. The system captures video of a golfer's swing, reconstructs a dynamic 3D model of the body, and classifies the swing to generate targeted coaching advice. This 12-week project will demonstrate the technical feasibility of using geometric deep learning on human pose data for sports coaching applications.

\section{Problem Statement}

Golf instruction traditionally requires expensive one-on-one sessions with professional coaches. While video analysis apps exist, they rely on 2D landmark detection and lack the geometric understanding needed for precise biomechanical feedback. Golfers need accessible, accurate, and actionable coaching that understands their swing in three dimensions.

\section{Technical Approach}

\subsection{3D Human Pose Estimation}

We employ HMR 2.0 (Human Mesh Recovery) to extract the SMPL body model from video frames.

\begin{figure}[h]
    \centering
    \begin{subfigure}[b]{0.48\textwidth}
        \centering
        \includegraphics[width=\textwidth]{figures/swing.jpg}
        \caption{Input image}
    \end{subfigure}
    \hfill
    \begin{subfigure}[b]{0.48\textwidth}
        \centering
        \includegraphics[width=\textwidth]{figures/swing_render.png}
        \caption{Reconstructed 3D mesh overlay}
    \end{subfigure}
    \caption{Our pipeline extracts a 3D human mesh from a single image, enabling geometric analysis of body pose.}
    \label{fig:pipeline}
\end{figure}

\subsubsection{Video Processing and Biomechanical Analysis}

We have extended our pipeline to process video sequences, enabling temporal analysis of the golf swing. Figure~\ref{fig:swing_sequence} shows five frames extracted from a recorded swing, capturing the key phases from top of backswing through finish.

\begin{figure}[h]
    \centering
    \begin{subfigure}[b]{0.18\textwidth}
        \centering
        \includegraphics[width=\textwidth]{figures/frame_1.png}
        \caption{}
    \end{subfigure}
    \hfill
    \begin{subfigure}[b]{0.18\textwidth}
        \centering
        \includegraphics[width=\textwidth]{figures/frame_2.png}
        \caption{}
    \end{subfigure}
    \hfill
    \begin{subfigure}[b]{0.18\textwidth}
        \centering
        \includegraphics[width=\textwidth]{figures/frame_3.png}
        \caption{}
    \end{subfigure}
    \hfill
    \begin{subfigure}[b]{0.18\textwidth}
        \centering
        \includegraphics[width=\textwidth]{figures/frame_4.png}
        \caption{}
    \end{subfigure}
    \hfill
    \begin{subfigure}[b]{0.18\textwidth}
        \centering
        \includegraphics[width=\textwidth]{figures/frame_5.png}
        \caption{}
    \end{subfigure}
    \caption{Key phases of a golf swing from top of backswing through finish (33 frames at 30 fps).}
    \label{fig:swing_sequence}
\end{figure}

From each frame, we extract the SMPL body model and compute joint angles. Figure~\ref{fig:elbow_angles} demonstrates our ability to track biomechanical parameters over time, showing the left and right elbow bend angles throughout the swing. The smooth curves indicate reliable pose estimation, and the characteristic patterns---left arm extending while the right arm bends through impact---align with established golf biomechanics.

\begin{figure}[h]
    \centering
    \includegraphics[width=0.85\textwidth]{figures/elbow_angles.png}
    \caption{Elbow angles extracted from video analysis. Raw measurements (dots) and Savitzky-Golay smoothed curves show consistent tracking of arm mechanics through the swing.}
    \label{fig:elbow_angles}
\end{figure}

\subsection{Geometric Representation of Pose}

The pose parameter $\boldsymbol{\theta}$ consists of 23 rotation matrices $\{R_i\}_{i=1}^{23}$, where each $R_i \in SO(3)$ represents the relative rotation of joint $i$ with respect to its parent in the kinematic tree. A complete body pose is thus an element of the product manifold:

\begin{equation}
    \boldsymbol{\theta} \in SO(3)^{23}
\end{equation}

A golf swing, captured as a sequence of frames, becomes a smooth curve:

\begin{equation}
    \gamma: [T_0, T_1] \rightarrow SO(3)^{23}
\end{equation}

with $T_0 < 0 < T_1$ and $t=0$ the moment of ball contact. This geometric perspective enables us to leverage tools from Riemannian geometry and Lie group theory for swing analysis---computing geodesic distances between poses, analyzing angular velocities, and identifying key swing phases.

\subsection{Swing Alignment and Normalization}

We focus on the \textbf{drive} (full swing), as it has the most consistent and analyzable structure. To enable comparison across golfers, we apply two forms of normalization:

\textbf{Temporal alignment.} We define:
\begin{itemize}[noitemsep]
    \item $T_0$ = top of backswing (detected as the reversal point in shoulder/hip rotation)
    \item $t = 0$ = moment of ball contact
    \item $T_1$ = finish position (motion stops)
\end{itemize}

\textbf{Spatial normalization.} The SMPL body pose representation is already well-suited for comparison: each $R_i \in SO(3)$ is a \emph{relative} rotation with respect to the parent joint. This means two golfers of different heights with identical swing mechanics will have the same body pose values---body shape differences are factored into separate shape parameters $\boldsymbol{\beta}$, not the pose.

\textbf{Fixed-frame resampling.} Swing faults are fundamentally \emph{geometric}---they concern the shape of the curve through $SO(3)^{23}$, not the speed at which it is traversed. A fast swing and a slow swing with identical body positions should receive the same feedback. We therefore resample all swings to a fixed number of frames $N$ (e.g., $N=30$), yielding a tensor of shape $(N, 23, 3, 3)$ regardless of original tempo.

\subsection{Swing Classification}

Our goal is to train a classifier $f: \mathcal{C} \rightarrow \mathcal{Y}$ that maps swing curves $\mathcal{C}$ to feedback categories $\mathcal{Y}$. Categories may include:

\begin{itemize}[noitemsep]
    \item Swing plane deviations (over-the-top, too flat)
    \item Hip rotation timing (early extension, restricted turn)
    \item Weight transfer patterns (reverse pivot, sway)
    \item Arm/wrist mechanics (casting, flipping)
\end{itemize}

The classifier will be developed through:
\begin{enumerate}[noitemsep]
    \item \textbf{Data collection}: Gathering labeled swing videos with expert annotations
    \item \textbf{Prior incorporation}: Encoding biomechanical rules from golf instruction literature
    \item \textbf{Expert collaboration}: Iterative refinement with PGA-certified instructors
\end{enumerate}

\end{document}
